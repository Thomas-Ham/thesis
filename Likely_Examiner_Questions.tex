\chapter{Likely Examiner Questions}\label{Likely Examiner Questions}

After having read parts of the thesis, these are some of the questions people think are likely to come up in the viva. Don't necessarily need to include an explanation in the text, but should be prepared to answer.

(Also stuff I should probably know but could do with checking / reading up on.)

\begin{itemize}
    \item Background/Theory
    \begin{itemize}
        \item Brief description / be able to sketch the historically significant neutrino experiments.
        \item Benefits of on-axis vs off-axis detectors. https://arxiv.org/pdf/hep-ex/0111033.pdf
        \item Exclusion contour vs exclusion sensitivity contour? STEREO plot.
        \item Wilks' theorem - what is it? And why does it apply/maybe not apply to neutrino experiments.
    \end{itemize}
    
    \item SBN Programme
    \begin{itemize}
        \item Energy threshold calculation for nue, numu, nutau interaction. (see https://www.hep.phy.cam.ac.uk/~thomson/partIIIparticles/handouts/Handout\_11\_2011.pdf)
    \end{itemize}
    
    \item Energy Reco
    \begin{itemize}
        \item Why the deconvolved charge drops below 0? Due to FFT when extracting the raw signal from the convolutional integral.
        \item Explanation for the resolution comparisons. Linear and ESTAR pretty close - I think this may just be a binning artefact. Num Electrons a little wider - it systematically assigns higher energies to the hits, so the spread in energies will also be a little broader hence the wider fit. 
    \end{itemize}
    
    \item Input Osc Params
    \begin{itemize}
        \item Selection - beam spill time and bucket structure.
    \end{itemize}
    
    \item VALOR
    \begin{itemize}
        \item Understand the origin of the dominant systematics + how they are modelled.
        \item Cross-section uncertainties are modelled on hydrogen and deuterium data - how/why is this relevant for argon? Only considers \numu - does this map to \nue?  
        \item Why do exclusive samples help. 
    \end{itemize}
\end{itemize}
