\usepackage{xspace}
\usepackage{tikz}
\usepackage{morefloats,afterpage}
\usepackage{mathrsfs} % script font
\usepackage{verbatim}
\usepackage{multirow}
\usepackage{nicematrix}
\usepackage{feynmf} % feynman diagrams
\usepackage{adjustbox}
\usepackage{arydshln}
\usepackage{subcaption}

\usepackage[utf8]{inputenc}
\usepackage{amsmath}
\usepackage{amsfonts}
\usepackage{amssymb}
\usepackage{fullpage}
\usepackage{lscape}

\usepackage[a4paper,
            left=30mm,
            right=20mm]{geometry}

% glossary - add to toc and remove page numbers
\usepackage[xindy,toc,nonumberlist]{glossaries} 
%\newglossaryentry{domain-knowledge}{%
%  name={domain knowledge},%
%  description={valid knowledge used to refer to an area of human endeavour, an autonomous computer activity, or other specialized discipline}}

\newacronym{adc}{ADC}{Analog Digital Converter}
\newacronym{apa}{APA}{Andode Plane Assembly}
\newacronym{arapuca}{ARAPUCA}{Argon R\&D Advanced Program at UniCAmp}
\newacronym{argoneut}{ArgoNeuT}{Argon Neutrino Test Stand}
\newacronym{bnb}{BNB}{Booster Neutrino Beam}
\newacronym{bsm}{BSM}{Beyond Standard Model}
\newacronym{cc}{CC}{Charged Current}
\newacronym{ccqe}{CCQE}{Charged Current Quasi Elastic}
\newacronym{cp}{CP}{charge-parity}
\newacronym{cpa}{CPA}{Cathode Plane Assembly}
\newacronym{crt}{CRT}{Cosmic Ray Tagger}
\newacronym{donut}{DONUT}{Direct Observation of the Nu Tau}
\newacronym{dune}{DUNE}{Deep Underground Neutrino Experiment}
\newacronym{em}{EM}{electromagnetic}
\newacronym{es}{ES}{Elastic Scattering}
\newacronym{fermilab}{Fermilab}{Fermi National Accelerator Laboratory}
\newacronym{fnal}{FNAL}{Fermi National Accelerator Laboratory}
\newacronym{gallex}{GALLEX}{Gallium Experiment}
\newacronym{icarus}{ICARUS}{Imaging Cosmic and Rare Underground Signals}
\newacronym{karmen}{KARMEN}{Karlsruhe Rutherford Medium Energy Neutrino}
\newacronym{larsoft}{LArSoft}{Liquid Argon Software}
\newacronym{lartpc}{LArTPC}{Liquid Argon Time Projection Chamber}
\newacronym{lep}{LEP}{Large Electron-Positron Collider}
\newacronym{lsnd}{LSND}{Liquid Scintillator Neutrino Detector}
\newacronym{mc}{MC}{Monte Carlo}
\newacronym{microboone}{MicroBooNE}{Micro Booster Neutrino Experiment}
\newacronym{miniboone}{MiniBooNE}{Mini Booster Neutrino Experiment}
\newacronym{minos}{MINOS}{Main Injector Neutrino Oscillation Search}
\newacronym{mip}{MIP}{Minimum Ionising Particle}
\newacronym{nc}{NC}{Neutral Current}
\newacronym{nist}{NIST}{National Institute of Standards and Technology}
\newacronym{p}{P}{parity}
\newacronym{pdg}{PDG}{Particle Data Group}
\newacronym{pds}{PDS}{Photon Detection System}
\newacronym{pmns}{PMNS}{Pontecorvo-Maki-Nakagawa-Sakata}
\newacronym{pmt}{PMT}{Photo Multiplier Tube}
\newacronym{sage}{SAGE}{Soviet-American Gallium Experiment}
\newacronym{sbn}{SBN}{Short Baseline Neutrino}
\newacronym{sbnd}{SBND}{Short Baseline Near Detector}
\newacronym{sce}{SCE}{Space Charge Effect}
\newacronym{sipm}{SiPM}{Silicon Photomultiplier}
\newacronym{sk}{SK}{Super Kamiokande}
\newacronym{sm}{SM}{Standard Model}
\newacronym{sno}{SNO}{Sudbury Neutrino Observatory}
\newacronym{sp}{SP}{Space Point}
\newacronym{tla}{TLA}{Three Letter Acronym}
\newacronym{tpb}{TPB}{Tetraphenyl Butadiene}
\newacronym{tpc}{TPC}{Time Projection Chamber}
\newacronym{vuv}{VUV}{Vacuum Ultra Violet}





\makeglossaries

% sort out the captions
\captionsetup{width = \textwidth}
\captionsetup{format = plain} %

% Define change margin command
\def\changemargin#1#2{\list{}{\rightmargin#2\leftmargin#1}\item[]}
\let\endchangemargin=\endlist 

%% Using Babel allows other languages to be used and mixed-in easily
%\usepackage[ngerman,english]{babel}
\usepackage[english]{babel}
\selectlanguage{english}

%% Citation system tweaks
\usepackage{cite}
% \let\@OldCite\cite
% \renewcommand{\cite}[1]{\mbox{\!\!\!\@OldCite{#1}}}

%% Maths
% TODO: rework or eliminate maybemath
\usepackage{abmath}
\DeclareRobustCommand{\mymath}[1]{\ensuremath{\maybebmsf{#1}}}
% \DeclareRobustCommand{\parenths}[1]{\mymath{\left({#1}\right)}\xspace}
% \DeclareRobustCommand{\braces}[1]{\mymath{\left\{{#1}\right\}}\xspace}
% \DeclareRobustCommand{\angles}[1]{\mymath{\left\langle{#1}\right\rangle}\xspace}
% \DeclareRobustCommand{\sqbracs}[1]{\mymath{\left[{#1}\right]}\xspace}
% \DeclareRobustCommand{\mods}[1]{\mymath{\left\lvert{#1}\right\rvert}\xspace}
% \DeclareRobustCommand{\modsq}[1]{\mymath{\mods{#1}^2}\xspace}
% \DeclareRobustCommand{\dblmods}[1]{\mymath{\left\lVert{#1}\right\rVert}\xspace}
% \DeclareRobustCommand{\expOf}[1]{\mymath{\exp{\!\parenths{#1}}}\xspace}
% \DeclareRobustCommand{\eexp}[1]{\mymath{e^{#1}}\xspace}
% \DeclareRobustCommand{\plusquad}{\mymath{\oplus}\xspace}
% \DeclareRobustCommand{\logOf}[1]{\mymath{\log\!\parenths{#1}}\xspace}
% \DeclareRobustCommand{\lnOf}[1]{\mymath{\ln\!\parenths{#1}}\xspace}
% \DeclareRobustCommand{\ofOrder}[1]{\mymath{\mathcal{O}\parenths{#1}}\xspace}
% \DeclareRobustCommand{\SOgroup}[1]{\mymath{\mathup{SO}\parenths{#1}}\xspace}
% \DeclareRobustCommand{\SUgroup}[1]{\mymath{\mathup{SU}\parenths{#1}}\xspace}
% \DeclareRobustCommand{\Ugroup}[1]{\mymath{\mathup{U}\parenths{#1}}\xspace}
% \DeclareRobustCommand{\I}[1]{\mymath{\mathrm{i}}\xspace}
% \DeclareRobustCommand{\colvector}[1]{\mymath{\begin{pmatrix}#1\end{pmatrix}}\xspace}
\DeclareRobustCommand{\Rate}{\mymath{\Gamma}\xspace}
\DeclareRobustCommand{\RateOf}[1]{\mymath{\Gamma}\parenths{#1}\xspace}

%% High-energy physics stuff
\usepackage{abhep}
\usepackage{hepnames}
\usepackage{hepunits}
\DeclareRobustCommand{\arXivCode}[1]{arXiv:#1}
\DeclareRobustCommand{\CP}{\ensuremath{\mathcal{CP}}\xspace}
\DeclareRobustCommand{\CPviolation}{\CP-violation\xspace}
\DeclareRobustCommand{\CPv}{\CPviolation}
\DeclareRobustCommand{\LHCb}{LHCb\xspace}
\DeclareRobustCommand{\LHC}{LHC\xspace}
\DeclareRobustCommand{\LEP}{LEP\xspace}
\DeclareRobustCommand{\CERN}{CERN\xspace}
\DeclareRobustCommand{\bphysics}{\Pbottom-physics\xspace}
\DeclareRobustCommand{\bhadron}{\Pbottom-hadron\xspace}
\DeclareRobustCommand{\Bmeson}{\PB-meson\xspace}
\DeclareRobustCommand{\bbaryon}{\Pbottom-baryon\xspace}
\DeclareRobustCommand{\Bdecay}{\PB-decay\xspace}
\DeclareRobustCommand{\bdecay}{\Pbottom-decay\xspace}
\DeclareRobustCommand{\BToKPi}{\HepProcess{ \PB \to \PK \Ppi }\xspace}
\DeclareRobustCommand{\BToPiPi}{\HepProcess{ \PB \to \Ppi \Ppi }\xspace}
\DeclareRobustCommand{\BToKK}{\HepProcess{ \PB \to \PK \PK }\xspace}
\DeclareRobustCommand{\BToRhoPi}{\HepProcess{ \PB \to \Prho \Ppi }\xspace}
\DeclareRobustCommand{\BToRhoRho}{\HepProcess{ \PB \to \Prho \Prho }\xspace}
\DeclareRobustCommand{\X}{\thesismath{X}\xspace}
\DeclareRobustCommand{\Xbar}{\thesismath{\overline{X}}\xspace}
\DeclareRobustCommand{\Xzero}{\HepGenParticle{X}{}{0}\xspace}
\DeclareRobustCommand{\Xzerobar}{\HepGenAntiParticle{X}{}{0}\xspace}
\DeclareRobustCommand{\epluseminus}{\Ppositron\!\Pelectron\xspace}
\DeclareRobustCommand{\protonproton}{\Pproton\APantiproton\xspace}
